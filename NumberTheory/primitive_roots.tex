\documentclass{article}
\usepackage{amsmath}
\usepackage{amsfonts}
\title{Primitive Roots}
\author{Nicholas Wendt}
\date{November 18, 2025}


\begin{document}

\maketitle

\section{Recall}
Remember that $\mathbb{Z}_n^*$ is a set containing all numbers less than n that are relatively prime to n. In other words, $a \in \mathbb{Z}_n^*$ if $gcd(a,n)=1$.
\section{Euler's Theorem}
For any element $a \in \mathbb{Z}_n^*$, consider the sequence of powers of a: $1,a,a^2,a^3,...\ (mod\ n)$. Since the set $\mathbb{Z}_N^*$ is closed under multiplication, all these powers of $a\ (mod\ n)$ are also in the set.\\
It follows, by Euler's theorem, that the sequence of powers eventually repeats.
$$
a^0 \equiv 1 \equiv a^{\phi (n)}\ (mod\ n)
$$
This is a characteristic of primitive roots. Primitive roots are elements in $\mathbb{Z}_n^*$ such that the sequence of powers has a minimum period of $\phi (n)$. \\
\\
This minimum period is known as the \textbf{order} of a. For all primitive roots $a \in \mathbb{Z}_n^*$, the order of a is $\phi (n)$.\\

For any primitive root $g (mod n)$, the elements i 
\subsection{Existance of Primitive Roots}
Primitive roots exists mod n if and only if n is of the form $2,4,p^k,2p^k$ for any odd prime p.

\section{Tricks}
If you have a primitive root $g\ (mod\ n)$ then any other element $g^r$ where $gcd(r,\phi(n))=1$ is also a primitive root. \\
If you suspect $g$ is a primitve root $mod\ p$ then you check $g^{\frac{(p-1)}{q}} \not\equiv 1\ (mod\ p)$ for every prime divisor, $q$, of $p-1$. If none of these powers of $g$ are equivilant to $1\ (mod\ p)$ then $g$ is a primitive root.\\
Once you know one primitive root, $g$, then all other primitive roots are of the form $g^r$ where $gcd(r,\phi(n))=1$.



\subsection{Example}
\end{document}

