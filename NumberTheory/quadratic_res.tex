\documentclass{article}

\title{Quadratic Residues}
\author{Nicholas Wendt}
\date{November 18, 2025}

\usepackage{amsmath}
\usepackage{amsfonts}

\begin{document}

\maketitle

\section{Intro}
Some integer a is a quadratic residue mod n if there exists some integer x such that:
$$
x^2 \equiv a\ (mod\ n)
$$

If no x exists, then a is called a quadratic nonresidue mod n.
\section{Legendre Symbol}
\[
\left( \frac{a}{p} \right) =
\begin{cases}
  1 & \text{if } a \text{ is a quadratic residue mod } p, \\[6pt]
 -1 & \text{if } a \text{ is a nonresidue mod } p, \\[6pt]
  0 & \text{if } p \mid a.
\end{cases}
\]
\subsection{Euler's Criterion}
For an odd prime p and integer a not divisible by p,
\[
a^{\frac{p-1}{2}} \equiv
\begin{cases}
  1 \pmod{p}, & \text{if } a \text{ is a quadratic residue}, \\[6pt]
 -1 \pmod{p}, & \text{if } a \text{ is a nonresidue.}
\end{cases}
\]
\subsection{Monday Rule}
\[
\left( \frac{2}{p} \right) =
\begin{cases}
  1, & \text{if } p \equiv 1 \text{ or } 7 \pmod{8}, \\[6pt]
 -1, & \text{if } p \equiv 3 \text{ or } 5 \pmod{8}.
\end{cases}
\]

\section{Quadratic Reciprocity}
\[
\textbf{Law of Quadratic Reciprocity:} \quad
\left( \frac{p}{q} \right)
\left( \frac{q}{p} \right)
= (-1)^{\frac{(p-1)(q-1)}{4}},
\]
where \(p\) and \(q\) are distinct odd primes. 

\[
\text{Equivalently,} \quad
\left( \frac{p}{q} \right) =
\begin{cases}
  \left( \frac{q}{p} \right), & \text{if } p \equiv 1 \text{ or } q \equiv 1 \pmod{4}, \\[6pt]
 -\left( \frac{q}{p} \right), & \text{if } p \equiv q \equiv 3 \pmod{4}.
\end{cases}
\]

\end{document}
